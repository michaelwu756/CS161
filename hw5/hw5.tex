\documentclass[12pt]{article}
\usepackage{amsmath}
\begin{document}
\title{Computer Science 161, Homework 5}
\date{June 5th, 2018}
\author{Michael Wu\\UID: 404751542}
\maketitle

\section*{Problem 1}

\paragraph{a)}

Neither. Using inference rules this statement \(S\) becomes the following.
\[S=(\text{Smoke}\land\neg\text{Fire})\lor\text{Smoke}\lor\neg\text{Fire}\]
The following truth table describes the possible worlds that this statement can apply to.
\begin{center}
        \begin{tabular}{c|c|c}
                Smoke & Fire & \(S\)\\
                \hline
                False & False & True\\
                False & True & False\\
                True & False & True\\
                True & True & True
        \end{tabular}
\end{center}
This statement is true in some worlds and false in others, thus it is neither valid nor unsatisfiable.

\paragraph{b)}

Neither. Using inference rules this statement \(S\) becomes the following.
\[S=(\text{Smoke}\land\neg\text{Fire})\lor (\neg\text{Smoke}\land\neg\text{Heat}) \lor \text{Fire}\]
The following truth table describes the possible worlds that this statement can apply to.
\begin{center}
        \begin{tabular}{c|c|c|c}
                Smoke & Fire & Heat & \(S\)\\
                \hline
                False & False & False & True\\
                False & False & True & False\\
                False & True & False & True\\
                False & True & True & True\\
                True & False & False & True\\
                True & False & True & True\\
                True & True & False & True\\
                True & True & True & True
        \end{tabular}
\end{center}
This statement is true in some worlds and false in others, thus it is neither valid nor unsatisfiable.

\paragraph{c)}

Valid. Using inference rules this statement \(S\) becomes the following.
\[S=(\neg\text{Smoke}\lor\neg\text{Heat}\lor\text{Fire})\lor(\text{Smoke}\land\text{Heat}\land\neg\text{Fire})\]
The following truth table describes the possible worlds that this statement can apply to.
\begin{center}
        \begin{tabular}{c|c|c|c}
                Smoke & Fire & Heat & \(S\)\\
                \hline
                False & False & False & True\\
                False & False & True & True\\
                False & True & False & True\\
                False & True & True & True\\
                True & False & False & True\\
                True & False & True & True\\
                True & True & False & True\\
                True & True & True & True
        \end{tabular}
\end{center}
This statement is true in all worlds, thus it is valid.

\section*{Problem 2}

\paragraph{a)}

Let the following be our variables.
\begin{align*}
        A &= \text{The unicorn is mythical.}\\
        B &= \text{The unicorn is mortal.}\\
        C &= \text{The unicorn is a mammal.}\\
        D &= \text{The unicorn is horned.}\\
        E &= \text{The unicorn is magical.}
\end{align*}
Then our knowledge base is the following
\begin{align*}
        A&\implies \neg B\\
        \neg A &\implies B \land C\\
        \neg B \lor C &\implies D\\
        D &\implies E
\end{align*}

\paragraph{b)}

\[(\neg A \lor \neg B)\land(A \lor B)\land(A \lor C)\land(B \lor D)\land(\neg C \lor D)\land(\neg D \lor E)\]

\paragraph{c)}

We will first use resolution to see if we can prove \(A\). If our knowledge base is unsatisfiable when we add \(\neg A\), then we have proven \(A\).
\begin{gather*}
        (A \lor B)\land\neg A\implies B\\
        (A \lor C)\land\neg A\implies C\\
        (\neg C \lor D)\land C\implies D\\
        (\neg D \lor E)\land D\implies E
\end{gather*}
We have generated a world \((\neg A, B, C, D, E)\) which shows that the knowledge base is satisfiable if we add \(\neg A\),
so we cannot prove that the unicorn is mythical.

Next we will use resolution to see if we can prove \(E\). If our knowledge base is unsatisfiable when we add \(\neg E\), then we have proven \(E\).
\begin{gather*}
        (\neg D \lor E)\land\neg E\implies\neg D\\
        (\neg C \lor D)\land\neg D\implies\neg C\\
        (B \lor D)\land\neg D\implies B\\
        (A \lor C)\land\neg C\implies A\\
        (\neg A \lor \neg B)\land B\implies\neg A
\end{gather*}
We have \(A\) and \(\neg A\), so this knowledge base is unsatisfiable. Thus we have proven \(E\), and the unicorn must be magical.

Finally, we will use resolution to see if we can prove \(D\). If our knowledge base is unsatisfiable when we add \(\neg D\), then we have proven \(D\).
\begin{gather*}
        (B \lor D)\land\neg D\implies B\\
        (\neg C \lor D)\land\neg D\implies\neg C\\
        (\neg A \lor \neg B)\land B\implies\neg A\\
        (A \lor C)\land\neg A\implies C
\end{gather*}
We have \(\neg C\) and \(C\), so this knowledge base is unsatisfiable. Thus we have proven \(D\), and the unicorn must be horned.

\section*{Problem 3}

\paragraph{a)}

\[\{x/\text{A},y/\text{B},z/\text{B}\}\]

\paragraph{b)}

No unification exists. When trying to unify, we would set \(y\) to \(G(x,x)\), but then this means that we would try to set \(x\) to \(A\) and
\(x\) to \(B\) at the same time.

\paragraph{c)}

\[\{y/\text{John}, x/\text{John}\}\]

\paragraph{d)}

No unification exists. When trying to unify, we would set \(x\) to \(\text{Father}(y)\) and \(y\) at the same time, which would be inconsistent.

\section*{Problem 4}

\paragraph{a)}

\begin{enumerate}
        \item \(\forall x\,(\text{Food}(x)\implies\text{Likes}(\text{John},x))\)
        \item \(\text{Food}(\text{Apples})\)
        \item \(\text{Food}(\text{Chicken})\)
        \item \(\forall x,y\,(\text{Eats}(x,y)\land\neg \text{KilledBy}(x,y)\implies\text{Food}(y))\)
        \item \(\forall x,y\,(\text{KilledBy}(x,y)\implies \neg \text{Alive}(x))\)
        \item \(\text{Eats}(\text{Bill},\text{Peanuts})\land \text{Alive}(\text{Bill})\)
        \item \(\forall x\,(\text{Eats}(\text{Bill},x)\implies\text{Eats}(\text{Sue},x))\)
\end{enumerate}

\paragraph{b)}

We have the following clauses.
\begin{enumerate}
        \item \(\neg\text{Food}(x)\lor\text{Likes}(\text{John},x)\)
        \item \(\text{Food}(\text{Apples})\)
        \item \(\text{Food}(\text{Chicken})\)
        \item \(\neg\text{Eats}(y,z)\lor \text{KilledBy}(y,z)\lor\text{Food}(z)\)
        \item \(\neg\text{KilledBy}(a,b)\lor \neg \text{Alive}(a)\)
        \item \(\text{Eats}(\text{Bill},\text{Peanuts})\)
        \item \(\text{Alive}(\text{Bill})\)
        \item \(\neg\text{Eats}(\text{Bill},c)\lor\text{Eats}(\text{Sue},c)\)
\end{enumerate}

\paragraph{c)}

Assume for contradiction that \(\neg\text{Likes}(\text{John},\text{Peanuts})\). Then we can apply resolution to generate a contradiction as shown
below.
\[(\neg\text{Food}(x)\lor\text{Likes}(\text{John},x))\land\neg\text{Likes}(\text{John},\text{Peanuts})\implies\neg\text{Food}(\text{Peanuts})\]
\begin{multline*}
        (\neg\text{Eats}(y,z)\lor \text{KilledBy}(y,z)\lor\text{Food}(z))\land\neg\text{Food}(\text{Peanuts})\implies\\
        \neg\text{Eats}(y,\text{Peanuts})\lor \text{KilledBy}(y,\text{Peanuts})
\end{multline*}
\begin{multline*}
        (\neg\text{Eats}(y,\text{Peanuts})\lor \text{KilledBy}(y,\text{Peanuts}))\land\text{Eats}(\text{Bill},\text{Peanuts})\implies\\
        \text{KilledBy}(\text{Bill},\text{Peanuts})
\end{multline*}
\[(\neg\text{KilledBy}(a,b)\lor \neg \text{Alive}(a))\land\text{KilledBy}(\text{Bill},\text{Peanuts})\implies\neg\text{Alive}(\text{Bill})\]
This is a contradiction because \(\text{Alive}(\text{Bill})\) is already in our knowledge base, and so our knowledge base is unsatisfiable if
\(\neg\text{Likes}(\text{John},\text{Peanuts})\). Thus we have proven that John likes peanuts.

\paragraph{d)}

\[(\neg\text{Eats}(\text{Bill},c)\lor\text{Eats}(\text{Sue},c))\land\text{Eats}(\text{Bill},\text{Peanuts})\implies\text{Eats}(\text{Sue},\text{Peanuts})\]
There are no other ways to prove that Sue eats anything else, so our only possible answer is that Sue eats peanuts.

\paragraph{e)}

The replacement clauses are as follows.
\begin{enumerate}
        \item \(\text{Eats}(d,e)\lor \text{Die}(d)\)
        \item \(\neg \text{Die}(f)\lor \neg \text{Alive}(f)\)
        \item \(\text{Alive}(\text{Bill})\)
\end{enumerate}
Then by resolution we can prove the following.
\[(\neg \text{Die}(f)\lor \neg \text{Alive}(f))\land \text{Alive}(\text{Bill})\implies \neg\text{Die}(\text{Bill})\]
\[(\text{Eats}(d,e)\lor \text{Die}(d))\land \neg\text{Die}(\text{Bill})\implies \text{Eats}(\text{Bill},e)\]
\[(\neg\text{Eats}(\text{Bill},c)\lor\text{Eats}(\text{Sue},c))\land\text{Eats}(\text{Bill},e)\implies \text{Eats}(\text{Sue},e)\]
So \(\exists e\,\text{Eats}(\text{Sue},e)\land\text{Eats}(\text{Bill},e)\), but we do not have information to determine what \(e\) could be.

\end{document}